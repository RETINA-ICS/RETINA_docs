% Options for packages loaded elsewhere
\PassOptionsToPackage{unicode}{hyperref}
\PassOptionsToPackage{hyphens}{url}
%
\documentclass[
]{article}
\usepackage{amsmath,amssymb}
\usepackage{iftex}
\ifPDFTeX
  \usepackage[T1]{fontenc}
  \usepackage[utf8]{inputenc}
  \usepackage{textcomp} % provide euro and other symbols
\else % if luatex or xetex
  \usepackage{unicode-math} % this also loads fontspec
  \defaultfontfeatures{Scale=MatchLowercase}
  \defaultfontfeatures[\rmfamily]{Ligatures=TeX,Scale=1}
\fi
\usepackage{lmodern}
\ifPDFTeX\else
  % xetex/luatex font selection
\fi
% Use upquote if available, for straight quotes in verbatim environments
\IfFileExists{upquote.sty}{\usepackage{upquote}}{}
\IfFileExists{microtype.sty}{% use microtype if available
  \usepackage[]{microtype}
  \UseMicrotypeSet[protrusion]{basicmath} % disable protrusion for tt fonts
}{}
\makeatletter
\@ifundefined{KOMAClassName}{% if non-KOMA class
  \IfFileExists{parskip.sty}{%
    \usepackage{parskip}
  }{% else
    \setlength{\parindent}{0pt}
    \setlength{\parskip}{6pt plus 2pt minus 1pt}}
}{% if KOMA class
  \KOMAoptions{parskip=half}}
\makeatother
\usepackage{xcolor}
\setlength{\emergencystretch}{3em} % prevent overfull lines
\providecommand{\tightlist}{%
  \setlength{\itemsep}{0pt}\setlength{\parskip}{0pt}}
\setcounter{secnumdepth}{-\maxdimen} % remove section numbering
\ifLuaTeX
  \usepackage{selnolig}  % disable illegal ligatures
\fi
\usepackage{bookmark}
\IfFileExists{xurl.sty}{\usepackage{xurl}}{} % add URL line breaks if available
\urlstyle{same}
\hypersetup{
  pdftitle={RETINA Documentation},
  hidelinks,
  pdfcreator={LaTeX via pandoc}}

\title{RETINA Documentation}
\author{}
\date{}

\begin{document}
\maketitle

\section{RETINA Documentation}\label{retina-documentation}

\subsection{What is the RETINA
project}\label{what-is-the-retina-project}

\subsubsection{Intro to the pieces of the RETINA
project}\label{intro-to-the-pieces-of-the-retina-project}

\subsection{What are the main pieces of code that make up the RETINA
project}\label{what-are-the-main-pieces-of-code-that-make-up-the-retina-project}

\begin{enumerate}
\def\labelenumi{\arabic{enumi}.}
\tightlist
\item
  The RETINA app

  \begin{itemize}
  \tightlist
  \item
    primarily php code
  \item
    collects the data from the user
  \item
    shows sensor data to the user
  \item
    shows model run results to the user
  \end{itemize}
\item
  The RETINA database

  \begin{itemize}
  \tightlist
  \item
    PSQL \textbf{data catalogue} dedicated to storing RETINA project
    specific information such

    \begin{itemize}
    \tightlist
    \item
      as user login information
    \item
      site details
    \item
      site management
    \item
      records of model runs triggered through the RETINA app
    \end{itemize}
  \item
    The data in the RETINA database is supposed to be separate from the
    BETY database (described below.)
  \end{itemize}
\item
  The PECAN code

  \begin{itemize}
  \tightlist
  \item
    A bundle of R packages that handle

    \begin{itemize}
    \tightlist
    \item
      processing and preparing the data collected from the app and other
      sources to

      \begin{itemize}
      \tightlist
      \item
        use it as model inputs or
      \item
        compare with model outputs
      \end{itemize}
    \item
      reading model outputs and preparing them for visualizations,
      comparison with collected data and further analysis
    \end{itemize}
  \item
    For each model that is used in the RETINA project, there is a
    specific R PEcAn package that contains (at minimum)

    \begin{itemize}
    \tightlist
    \item
      script for preparing climate data for the model
    \item
      script(s) for preparing configuration (i.e.~parameterization)
      files for the model
    \item
      script to execute the model
    \item
      script to read the output of the model and translate it from model
      specific format in to a standardized format so that it can be
      easily comparible with other model outputs
    \end{itemize}
  \end{itemize}
\item
  The RETINA R package

  \begin{itemize}
  \tightlist
  \item
    A series of helper scripts written in R that are specific to the
    RETINA project and JHI and thus shouldn't be merged back in with the
    pecan project
  \item
    These scripts complement the tasks listed more generally with the
    PECAN code.
  \end{itemize}
\item
  The BETY database

  \begin{itemize}
  \tightlist
  \item
    PSQL database.
  \item
    While BETY does house some data, it is better to think of it as a
    \textbf{metadatabase} that primarily holds information on the
    location and format of data
  \item
    The BETY database is used to track the full provinance of all data
    used when a model run is generated by the RETINA app including:

    \begin{itemize}
    \tightlist
    \item
      records of all raw data and processed data used for drivers,
      parameterization and benchmarking (for every step of processing)
    \item
      records of run settings and model specific parameterization for
      model runs
    \item
      records of model run status and output files
    \end{itemize}
  \end{itemize}
\end{enumerate}

\end{document}
